\documentclass{article}
\usepackage{amsmath}
\usepackage{amssymb}
\usepackage{esvect}
\usepackage[usenames, dvipsnames]{color}
\usepackage{fancyhdr}
\usepackage{hyperref}
\usepackage{pgfplots}
\usepackage{tikz}
\usepackage{geometry}
\usepackage[normalem]{ulem}

\geometry{letterpaper, portrait, margin=0.5in}
\pagestyle{fancy}

\fancyhf{} % clear all header fields
\renewcommand{\headrulewidth}{0pt}
\fancyfoot[LE,RO]{\thepage}           % page number in "outer" position of footer line
\fancyfoot[RE,LO]{\copyright\;aquarc 2025. \href{https://aquarc.org}{\underline{aquarc.org}}} % other info in "inner" position of footer line

\definecolor{myred1}{RGB}{255, 0, 0}
\definecolor{myyellow1}{RGB}{255, 255, 219}
\definecolor{mygreen1}{RGB}{0, 255, 0}
\definecolor{mygreen2}{RGB}{0, 126, 0}
\definecolor{myblue1}{RGB}{0, 0, 255}

\begin{document}

\fontsize{14}{16}\selectfont

% center the title
\begin{center}
    \textbf{\underline{Logistic Function}}
\end{center}


A Logistic function is modeled by the following derivative:

\begin{align*}
    \frac{dP}{dt}=kP(1-\frac{P}{m})
\end{align*}

$k$ represents the maximum rate of growth as a percentage. Therefore, $kP$ will represent the absolute growth rate. \\

$m$ represents the maximum population size. Therefore, $\frac{P}{m}$ will approach 1 and the derivative will approach 0 as we approach carrying capacity.



\end{document}
