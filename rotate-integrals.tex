\documentclass{article}
\usepackage{amsmath}
\usepackage{amssymb}
\usepackage{esvect}
\usepackage[usenames, dvipsnames]{color}
\usepackage{fancyhdr}
\usepackage{hyperref}
\usepackage{pgfplots}
\usepackage{tikz}
\usepackage{geometry}
\usepackage[normalem]{ulem}

\geometry{letterpaper, portrait, margin=0.5in}
\pagestyle{fancy}

\fancyhf{} % clear all header fields
\renewcommand{\headrulewidth}{0pt}
\fancyfoot[LE,RO]{\thepage}           % page number in "outer" position of footer line
\fancyfoot[RE,LO]{\copyright\;aquarc 2025. \href{https://aquarc.org}{\underline{aquarc.org}}} % other info in "inner" position of footer line

\definecolor{myred1}{RGB}{255, 0, 0}
\definecolor{myyellow1}{RGB}{255, 255, 219}
\definecolor{mygreen1}{RGB}{0, 255, 0}
\definecolor{mygreen2}{RGB}{0, 126, 0}
\definecolor{myblue1}{RGB}{0, 0, 255}

\begin{document}

\fontsize{14}{16}\selectfont

% center the title
\begin{center}
    \textbf{\underline{Rotate Integrals Cheatsheet}}
\end{center}

\section{Disk Method}

When you have a function $f(x)$ you can rotate it around the x-axis like this:

\begin{align*}
    \pi\int_{a}^{b} (f(x))^2 dx
\end{align*}

Essentially, you are adding up many $\pi r^2$ Areas multiplied by width, or $dx$. So you're adding up a bunch of tiny volumes.

\section{Washer Method}

Subtract two Volumes to rotate it around the x-axis.

\begin{tikzpicture}[domain=0:4]
    \draw[very thin,color=gray] (-0.1,-1.1) grid (3.9,3.9);
    \draw plot (\x,0.25*\x^2) node[right] {$f(x) =-\frac{1}{4}x^2$ + 2};
    \draw plot (\x,1) node[right] {$g(x)=1$};
\end{tikzpicture}

\section{Cross-Sections}

\subsection{Triangle Cross Section}

\end{document}
