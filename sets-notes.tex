\documentclass{article}
\usepackage{amsmath}
\usepackage{amssymb}
\usepackage{fancyhdr}
\usepackage{hyperref}
\usepackage{tikz}
\usepackage{geometry}

\geometry{letterpaper, portrait, margin=0.5in}
\pagestyle{fancy}

\fancyhf{} % clear all header fields
\renewcommand{\headrulewidth}{0pt}
\fancyfoot[LE,RO]{\thepage}           % page number in "outer" position of footer line
\fancyfoot[RE,LO]{\copyright\;aquarc 2025. \href{https://aquarc.org}{\underline{aquarc.org}}} % other info in "inner" position of footer line

\begin{document}

\fontsize{14}{16}\selectfont

% center the title
\begin{center}
    \textbf{\underline{Sets Cheatsheet}}
\end{center}

\tableofcontents
\pagebreak

\section{Geometric Series}
Geometric Series are a special type of Power Series that can be rewritten in the form

$$
f(x)=\frac{a_1}{1-r}=\sum_{n=0}^{\infty}{
    a_1*r^n
}
$$

To find the interval of convergence, just remember that $|r|< 1$

\subsection{Examples}
$$
f(x)=\frac{1}{2-x}, c=0 
\Longrightarrow f(x)=\frac{1}{2}*\frac{1}{1-\frac{x}{2}}
\Longrightarrow f(x)=\sum_{n=0}^{\infty}{\frac{1}{2}*(\frac{x}{2})^n}
$$
$$
\Longrightarrow f(x)=\sum_{n=0}^{\infty}{(\frac{1}{2})^{n+1}x^n}
$$
$$
|r|<1 
\Longrightarrow |\frac{x}{2}|<1
\Longrightarrow |x|<2
\Longrightarrow x \in (-2,2)
$$

If we change the center:
$$
f(x)=\frac{1}{2-x}, c=5 \Longrightarrow f(x)=\frac{1}{2-5-(x-5)} 
\Longrightarrow \frac{1}{-3-(x-5)}
$$
$$
\Longrightarrow -\frac{1}{3}*\frac{1}{1-\frac{-(x-5)}{3}}
\Longrightarrow \sum_{n=0}^{\infty}{-\frac{1}{3}*(\frac{-(x-5)}{3})^n}
$$
$$
\Longrightarrow \sum_{n=0}^{\infty}{(-\frac{1}{3})^{n+1}(-(x-5))^n}
$$
$$
|r|<1
\Longrightarrow r \in (-1,1)
\Longrightarrow \frac{-x+5}{3} \in (-1,1)
\Longrightarrow -(x-5) \in (-3,3)
\Longrightarrow x \in (2,8)
$$
 
Now let's completely change it up.

$$
f(x)=\frac{3}{2x-1}, c=0 
\Longrightarrow f(x)=-3*\frac{1}{1-2x}
\Longrightarrow *\sum_{n=0}^{\infty}{-3*(2x)^n}, c \in (-\frac{1}{2},\frac{1}{2})
$$
Let's move the center
$$
f(x)=\frac{3}{2x-1}, c=-3
\Longrightarrow f(x)=\frac{1}{2(x+3)-5-6}
\Longrightarrow f(x)=-\frac{1}{11}*\frac{1}{1-\frac{2}{11}(x+3)}
$$
$$
f(x)=\sum_{n=0}^{\infty}{-\frac{1}{11}*(\frac{2}{11}(x+3))^n}
$$
$$
\frac{2}{11}(x+3) \in (-1,1) 
\Longrightarrow (x+3) \in (-\frac{11}{2},\frac{11}{2}) 
\Longrightarrow x \in (-\frac{17}{2},\frac{5}{2})
$$

What if there is no $-x$? You can just do $-(-x)$. Remember: Basic Algebra will take you a long way.

$$
f(x)=\frac{3}{x+2}, c=0 \Longrightarrow f(x)=\frac{3}{2}*\frac{1}{1-\frac{1}{2}(-x)}
$$

What if your function looks a bit more complicated?
$$
f(x)=\frac{4x-7}{2x^2+3x-2}, c=0
\Longrightarrow f(x)=\frac{4x-7}{(2x-1)(x+2)}
\Longrightarrow f(x)=\frac{A}{2x-1}+\frac{B}{x+2}
$$
$$
A(x+2)+B(2x-1)=4x-7 
\Longrightarrow Ax+2Bx=4x, 2A-B=-7
\Longrightarrow B=2A+7 
$$
$$
\Longrightarrow A+4A+14=4 
\Longrightarrow 5A=-10 
\Longrightarrow A=-2, B=3
$$
$$
f(x)=\frac{-2}{2x-1}+\frac{3}{x+2}
$$
Solve it normally from here.

\section{General Power Series}
$$
f(x)=\sum_{n=0}^{\infty}{\frac{f^{(n)}(c)}{n!}(x-c)^n}
$$

\subsection{Derivation}
We take general power series to be something like the following:
$$
f(x)=1+x+2x^2+3x^3+...+nx^n+...
$$
or
$$
f(x)=1+x+\frac{1}{2}x^2+\frac{1}{3}x^3+...+\frac{1}{n}x^n+...
$$

So generally:
$$
f(x)=\sum_{n=0}^{\infty}{a_n(x-c)^n}
$$
Instead of $a_i$ because the coefficient changes for each element in the series. Let's expand this series:

$$
f(x)=a_0+c+a_1(x-c)+a_2(x-c)^2+a_3(x-c)^3+...+a_n(x-c)^n + ...
$$
Take its derivative:
$$
f'(x)=0+a_1+2a_2(x-c)+3a_3(x-c)^2+...+na_n(x-c)^{n-1}+...
$$
Notice how $f'(0)=a_1$.
$$
f''(x)=0+0+2a_2+6a_3(x-c)+...+n(n-1)a_n(x-c)^{n-2}+...
$$
Notice how $f''(0)=2a_2$ and $f'''(0)=6a_3$. In order to get the $a_n$ term, you just need to take $f^{(n)}(c)$ and divide it by $n!$ \\
Put it together, and you'll get the equation we started with.

\subsection{Examples}
$$
f(x)=e^x,c=0
\Longrightarrow f(x)=\sum_{n=0}^{\infty}{\frac{e^0}{n!}(x)^n}
\Longrightarrow f(x)=\sum_{n=0}^{\infty}{\frac{x^n}{n!}}
$$
$$
\Longrightarrow 1+x+\frac{1}{2}x^2+\frac{1}{3}x^3+...+\frac{1}{n}x^n+...
$$
Evaluate a geometric series using Taylor series
$$
f(x)=\frac{1}{1-2x},c=0
\Longrightarrow \frac{\frac{1}{1}}{0!}+\frac{\frac{d}{dx}[\frac{1}{1-2x}}{1!}+\frac{d^2}{dx^2}[\frac{1}{1-2x}]{2!}+\frac{d^3}{dx^3}[\frac{1}{1-2x}]{3!}+...
$$
$$
\Longrightarrow 1+\frac{2}{(1-2x)^2}+\frac{4}{(1-2x)^3}+\frac{6}{(1-2x)^4}+...
$$
% TBD

\subsection{Error Checking}
In an ideal world, you want to know how far off your estimates are. For alternating series, this process is pretty easy.

$$
f(x)=1-x+\frac{x}{2!}-\frac{x}{3!}+\frac{x}{4!}-\frac{x}{5!}+\frac{x}{6!}-\frac{1}{7!}+...+\frac{1}{n!}(-1)^n+...
$$
Let's choose the first four terms for our \textbf{Taylor Polynomial}, which will be represented by $P_n(x)$ where $n$ is the degree.
$$
P_4(x)=1-x+\frac{1}{2}x^2-\frac{1}{3}x^3
$$
Let's set the remainder terms to $R_5(x)$. We can call the error $E(x)$.

$$
E(x)=|P_4(x)-R_5(x)|
$$

Since you will never truly know $R_5(x)$ or any $R_n(x)$ for that matter, you will never know the true error. But you can set an upper bound on that error, which is pretty easy in alternating power series.\\

$$
E(x)\leq P_{n+1}(x) - P_{n}(x)
$$
A fancy way of saying, the term right after is the highest the eror can be. It's because you will never add back what you subtracted, always less.

\subsection{Lagrange Error Checking}
If your series \textbf{isn't alternating}, you can use Lagrange Error Checking. Although the proof is complicated, it's pretty simple
$$
E(x)\leq|\frac{f^{(n+1)}(z)}{(n+1)!}|(x-c)^{n+1}
$$
Typically $z=\max(x,c)$, but that won't always be the case.

% doesn't make sense cuz how can it just be the next term, but whatever
\end{document}
