\documentclass{article}
\usepackage{amsmath}
\usepackage{amssymb}
\usepackage{fancyhdr}
\usepackage{hyperref}
\usepackage{tikz}
\usepackage{geometry}

\geometry{letterpaper, portrait, margin=0.5in}
\pagestyle{fancy}

\fancyhf{} % clear all header fields
\renewcommand{\headrulewidth}{0pt}
\fancyfoot[LE,RO]{\thepage}           % page number in "outer" position of footer line
\fancyfoot[RE,LO]{\copyright\;aquarc 2024. \href{https://aquarc.org}{\underline{aquarc.org}}} % other info in "inner" position of footer line

\title{SAT Math Equations}

\begin{document}

\fontsize{14}{16}\selectfont

\maketitle

\tableofcontents
\pagebreak

\section{Quadratics}

A quadratic is a function in the form
$$
f(x) = ax^2 + bx + c
$$
Where ${\{a, b, c\}} \in \mathbb{R}$.

Or in the vertex form:
$$
f(x) = a(x-h)^2 + k
$$
Where ${\{a, h, k\}} \in \mathbb{R}$

Or in the roots form:
$$
f(x) = a(x-r_1)(x-r_2)
$$
Where ${\{a, r_1, r_2\}} \in \mathbb{R}$

\subsection{FOIL}
FOIL stands for First Outer Inner Last. Meaning, when you have
$$
f(x)=(a_1x-r_1)(a_2x-r_2)
$$

You get:
$$
a_1a_2x^2-a_1r_2x-a_2r_1x+r_1r_2 =a_1a_2x^2-(a_1r_1+a_2r_2)x+(r_1r_2)
$$

\subsection{Quadratic Formula}

\subsubsection{Derivation}

$$
ax^2+bx+c=0 \Longrightarrow 
ax^2+bx=-c \Longrightarrow 
\frac{1}{a}ax^2+\frac{b}{a}x=-\frac{c}{a} 
\Longrightarrow
x^2+\frac{b}{a}x=-\frac{c}{a} 
$$
Complete the square:
$$
x^2+\frac{b}{a}x+\frac{b^2}{4a^2}=-\frac{c}{a} + \frac{b^2}{4a^2} \Longrightarrow
(x+\frac{b}{2a})^2=-\frac{4ac}{4a^2} + \frac{b^2}{4a^2} \Longrightarrow
(x+\frac{b}{2a})^2=\frac{b^2-4ac}{4a^2} 
$$
$$
\Longrightarrow
\sqrt{(x+\frac{b}{2a})^2}=\frac{\pm\sqrt{b^2-4ac}}{2a} \Longrightarrow
x=\frac{-b\pm\sqrt{b^2-4ac}}{2a}
$$

\subsubsection{Alternatives}

In some scenarios it may be easier to use systems of equations.

Given a quadratic in the form:
$$
ax^2+bx+c=0
$$
We want to find $r_1$ and $r_2$ such that:
$$
r_1+r_2=b \quad r_1r_2=c
$$

You can easily subsitute around until you get the answer.
If $a \neq 1$ then it is difficult to use this formula.

% todo: it is possible but how 

\subsection{Discriminant}

The discriminant tells you whether a quadratic is going to have real or imaginary roots. \\
If we look at the quadratic formula for earlier:

$$
x=\frac{-b\pm\sqrt{b^2-4ac}}{2a}
$$
The term $\pm\sqrt{b^2-4ac}$ denotes the possibility of two solutions or one solution or none. \\
If $\sqrt{b^2-4ac} > 0$ then the $\pm$ would have an effect, making the two solutions $-\frac{-b\pm\sqrt{b^2-4ac}}{2a}$. \\
If $\sqrt{b^2-4ac} = 0$ then the $\pm$ would have no effect, making the single solution $-\frac{-b}{2a}$. \\
However, if $b^2-4ac < 0$ then the $\sqrt{}$ of it would be a complex number, meaning there would be \textbf{no real} solutions.


\subsection{Vertex}


\subsection{Factoring}
You can use the two equations to do it

\subsubsection{Aquarc Method}
The fancy X thing


\section{Sets}

\subsection{Domain and Range}

\subsection{Standard Deviation}

\section{Word Problems}

You can't really predict what these will be. For 

\section{Trigonometry}


\end{document}
